\documentclass{article}

% Language setting
% Replace `english' with e.g. `spanish' to change the document language
\usepackage[english]{babel}

% Set page size and margins
% Replace `letterpaper' with `a4paper' for UK/EU standard size
\usepackage[a4paper,top=2cm,bottom=2cm,left=3cm,right=3cm,marginparwidth=1.75cm]{geometry}

% Useful packages
\usepackage{amsmath}
\usepackage{amsthm}
\usepackage{amsfonts}
\usepackage{amssymb}
\usepackage{geometry}
\usepackage{graphicx}
\usepackage{calc}
\usepackage{xassoccnt}
\usepackage{fontenc}
\usepackage{mathtools}
\usepackage{tikz}
\usepackage{caption}
\usepackage{enumerate}
\usepackage{xcolor}
\usepackage{sectsty}
\usepackage{microtype}
\usepackage{titlesec}
\usepackage{etoolbox}
\usepackage{chngcntr}
\usepackage[inline]{enumitem}
\usepackage[colorlinks=true, allcolors=blue]{hyperref}

\renewcommand{\qedsymbol}{\ensuremath{\blacksquare}}

%\newtheorem
\theoremstyle{definition}
\newtheorem{definition}{Definition}[subsection]

\newtheorem{theorem}{Theorem}[subsection]
\newtheorem{lemma}{Lemma}[subsection]
\newtheorem{corollary}{Corollary}[subsection]

\newtheorem{example}{Example}[section]
\newtheorem{exercise}{Exercise}
\newtheorem{question}{Question}

\DeclareCoupledCountersGroup{theorems}
\DeclareCoupledCounters[name=theorems]{theorem,lemma}

\theoremstyle{remark}
\newtheorem{remark}{Remark}[section]
\newtheorem*{solution}{Solution}

%\newcommands
\newcommand{\R}{\mathbb{R}}
\newcommand{\C}{\mathbb{C}}
\newcommand{\Z}{\mathbb{Z}}
\newcommand{\Q}{\mathbb{Q}}
\newcommand{\N}{\mathbb{N}}
\newcommand{\F}{\mathbb{F}}
\newcommand{\calP}{\mathcal{P}}
\newcommand{\powerset}[1]{\mathscr{P}(#1)}
\newcommand{\Mod}[1]{\ (\mathrm{mod}\ #1)}
%%%

\title{Ring Theory Midterm}
\author{Ata Berk Saraç}
\date{April 2025}

\begin{document}

\maketitle

\begin{question}
    For the ring $R=\Z_{60}$, determine:
    \begin{enumerate}[label=\alph*)]
        \item The set of zero-divisor elements.
        \item The set of nilpotent elements.
        \item The set of unit elements.
        \item The set of idempotent elements.
    \end{enumerate}
\end{question}

\begin{solution}
    The set of zero-divisor elements $Z(R)$ consists exactly of the elements $z\in R$ such that $\gcd(z,60)\ne 1$.

    The set of nilpotent elements $N(R)$ consists exactly of the elements $n\in R$ such that $\gcd(n,60)=2\cdot 3\cdot 5=30$.

    The set of unit elements $U(R)$ consists exactly of the elements $u\in R$ such that $\gcd(u,60)=1$.

    The set of idempotent elements $I(R)$ consists exactly of the elements $e\in R$ such that $e^2-e=e(e-1)\equiv 0 \Mod{60}$.

    The rest is just computation.
    \begin{align*}
        Z(R)=\{0,2,3,4,5,6,8,9,&10,12,14,15,16,18,20,21,22,24,25,26,27,28,30,32,33,34,\\
        &35,36,38,39,40,42,44,45,46,48,50,51,52,54,55,56,57,58\},
    \end{align*}
    $$N(R)=\{0,30\},$$
    $$U(R)=\{1,7,11,13,17,19,23,29,31,37,41,43,47,49,53,59\},$$
    $$I(R)=\{0,1,16,21,25,36,40,45\}.$$
\end{solution}

\begin{question}
    Prove that a non-trivial finite ring without non-zero zero divisors is a ring with identity.
\end{question}

\begin{solution}
    Before we start our proof, we sketch a proof for a claim that we will use. The claim is that if $S$ is a finite semigroup under multiplication, then $a^i=a^{i+p}$ for some minimal $i$ and $p$ for every $a\in S$. So, consider an infinite sequence $a, a^2, a^3, \ldots$, since $S$ is closed and finite, our result follows by the pigeonhole principle.
    
    Now, let $R$ be a non-trivial finite ring without non-zero zero divisors.

    We are trying to find an element $1$ in $R$ such that $x1=1x=x$ for all $x\in R$.

    For non-zero $a\in R$ and all $b,c\in R$, if $ab=ac$, we have $ab-ac=a(b-c)=0$ implies that $b=c$, a similar approach follows for $ba=ca$. Thus, right/left cancellation property holds in $R$.

    Fix a non-zero element $e$ of $R$, and assume that $e^i=e^{i+n}$ for some minimal $i$ and $n$ (since we treat $R$ as a finite semigroup under multiplication). If $x\mapsto xe$ is a mapping called $f$, then $f$ is injective since whenever $xe=ye$, we have $x=y$ for all $x,y\in R$, by cancellation. Since $R$ is finite, $f$ is bijective. Note that additivity property holds for $f$ since $f(x)+f(y)=xe+ye=(x+y)e=f(x+y)$ for all $x,y\in R$. It follows that $f(e^i)=e^{i+1}$ and $f(e^{i+n})=e^{i+n+1}$ so that $f(e^{i+n})-f(e^i)=f(e^{i+n}-e^{i})=(e^{i+n}-e^i)e=0e=f(0)$, hence $e^{i+n+1}=e^{i+1}$ so that $e^{n+1}-e=0$ by cancellation.
    
    It implies that $0x=(e^{n+1}-e)x=e(e^n x-x)=0$, implying that $e^n x=x$, by a similar algebraic computation $x0=x(e^{n+1}-e)=(x e^n-x)e=0$ implying that $x e^n=x$ so that $e^n$ is the identity we are looking for, for all $x\in R$. We have completed our proof.
\end{solution}

\begin{question}
    Define a regular ring, and prove that $R=M_2(\R)$ is a regular ring.
\end{question}

\begin{solution}
    A \textit{\textbf{regular ring}} is a ring $S$ where for each $a\in S$, there exists $x\in S$ such that $a=axa$.

    For invertible $A\in R$, the result follows by a simple computation: $A=AXA$ if and only if $X=A^{-1}$.

    For any non-invertible matrix $A \in R$ of rank $r=0,1,2$, there exist invertible $U, V\in R$ such that $A=U\begin{pmatrix} I_r & 0 \\ 0 & 0 \end{pmatrix}V$. We examine the case $r=1$ only (since $r=0$ is trivial and $r=2$ implies that $A$ is invertible): if $A=\begin{pmatrix} x & y \\ ax & ay\end{pmatrix}$, for  all nonzero $x,y\in \R$ and some $a\in \R$, then $A=\begin{pmatrix} 1 & 0 \\ a & 1 \end{pmatrix}\begin{pmatrix} 1 & 0 \\ 0 & 0 \end{pmatrix}\begin{pmatrix} x & y \\ -x & y \end{pmatrix}$. We have found that for
    \[X=\begin{pmatrix} {1}/{2x} & -{1}/{2x} \\ {1}/{2y} & {1}/{2y} \end{pmatrix}\begin{pmatrix} 1 & 0 \\ -a & 1 \end{pmatrix},\]
    we have the required result.
    We are done.
\end{solution}

\begin{question}
    Show that $S=\left\{\begin{pmatrix}
        a & a\\ a & a
    \end{pmatrix}\mid a\in \R \right\}$
    is a subring of $M_2(\R)$ and find the unity element of $S$.
\end{question}

\begin{solution}
    Since $\begin{pmatrix}0 & 0\\ 0 & 0\end{pmatrix}\in S$, $S$ is nonempty.
    
    For all $a,b\in\R$, $a-b\in\R$ and so $\begin{pmatrix}a & a\\ a & a\end{pmatrix}-\begin{pmatrix}b & b\\ b & b\end{pmatrix}=\begin{pmatrix}a-b & a-b\\ a-b & a-b\end{pmatrix}\in S$.

    For all $a,b\in\R$, $2ab\in \R$ and so $\begin{pmatrix}a & a\\ a & a\end{pmatrix}\begin{pmatrix}b & b\\ b & b\end{pmatrix}=\begin{pmatrix}2ab & 2ab\\ 2ab & 2ab\end{pmatrix}$.

    Thus, $S$ is a subring of $R$.

    For all $a,b\in \R$, we have $\begin{pmatrix}a & a\\ a & a\end{pmatrix}\begin{pmatrix}1/2 & 1/2\\ 1/2 & 1/2\end{pmatrix}=\begin{pmatrix}a & a\\ a & a\end{pmatrix}$ for all $\begin{pmatrix}a & a\\ a & a\end{pmatrix}$ and so $\begin{pmatrix}1/2 & 1/2\\ 1/2 & 1/2\end{pmatrix}\in S$ is the identity element of $S$.
\end{solution}

\begin{question}
    Give an example of a subring of $R=\Z_5\times\Z_5$ that is not an ideal of $R$, and write all the ideals of $R$.
\end{question}

\begin{solution}
    Since 5 is prime, the only ideals of $\Z_5$ are the trivial ones. Thus, $\{0\}\times\{0\}$, $\{0\}\times\Z_5$, $\Z_5\times\{0\}$, $\Z_5\times\Z_5$.

    Here is an example of a subring of $R$ that is not an ideal: $2\Z_5\times \Z_5$.
\end{solution}

\begin{question}
    Consider the set of all rational numbers $\Q$, where the binary operations are defined as:
    $$a\oplus b=a+b-1,\qquad a\odot b=ab-(a+b)+2.$$
    \begin{enumerate}[label=\alph*)]
        \item Show that $R=(\Q,\oplus,\odot)$ is a ring.
        \item Is $R$ commutative?
        \item Is there an identity element in $R$?
        \item Is $R$ an integral domain?
    \end{enumerate}
\end{question}

\begin{solution}
    \begin{enumerate}[label=\alph*)]
        \item Clearly, $R$ inherits its additive closure and commutativity from $\Q$, $1\in\Q$ is the additive identity of $R$, and $-a+1\in\Q$ is the additive inverse of $a\in R$, lastly, for all $a,b,c\in R$, 
        $$(a\oplus b)\oplus c=(a+b-1) \oplus c=a+b-1+c-1=a+(b+c-1)-1=a+(b\oplus c)-1=a\oplus(b\oplus c).$$

        $R$ inherits its multiplicative closure from $\Q$, $R$ is associative under multiplication
        \begin{align*}
            (a\odot b)\odot c 
            &= (ab-(a+b)+2)\odot c\\ 
            &= (ab-(a+b)+2)c-((ab-(a+b)+2)+c)+2\\
            &= abc-ac-bc+2c-ab+a+b-2-c+2\\
            &= abc-ab-ac+2a-a-bc+b+c-2+2\\
            &= a(bc-(b+c)+2)-(a+(bc-(b+c)+2))+2\\
            &= a\odot (bc-(b+c)+2)\\
            &= a\odot (b\odot c)
        \end{align*}
        for all $a,b\in\Q$
        \item $R$ inherits its multiplicative commutativity from the additive and multiplicative commutativity of $\Q$.
        \item $a\odot b=ab-(a+b)+2=a$, then $ab-2a=a(b-2)=b-2$ for all $a\in \Q$, which means that $b=2$ is the identity of $R$.
        \item $a\odot b=ab-(a+b)+2=0$, then $a=\frac{2-b}{1-b}$, thus such natural number pairs $a,b\ne 1=0_R$ are zero-divisors of $R$.
    \end{enumerate}
\end{solution}

\begin{question}
    Let $R$ be a ring such that for all $x\in R$, $x^2+x$ is in $Z(R)$, the center of $R$. Prove that $R$ is commutative.
\end{question}

\begin{solution}
    We assume that for any $x \in R$, $x^2+x$ is in the center $Z(R)$ of the ring $R$. This means that for any $x, a \in R$, we have $(x^2+x)a = a(x^2+x)$.
    Expanding, we obtain that
    $$x^2a + xa = ax^2 + ax$$
    This implies that
    $$x^2a - ax^2 = ax - xa \quad (*)$$
    This equality holds for all $x, a \in R$.
    
    Consider the element $(a+b)^2 + (a+b)$ for any elements $a, b \in R$. By assumption, this element is in the center of $R$. 
    $$(a+b)^2 + (a+b) = a^2 + ab + ba + b^2 + a + b$$
    So, for all $c \in R$, we have:$$(a^2 + ab + ba + b^2 + a + b)c = c(a^2 + ab + ba + b^2 + a + b)$$
    Using the fact that $a^2+a \in Z(R)$ and $b^2+b \in Z(R)$, we have $a^2c+ac = ca^2+ca$ and $b^2c+bc = cb^2+cb$. Expanding the previous equality:
    $$a^2c + ac + bc + b^2c + abc + bac = ca^2 + ca + cb + cb^2 + cab + cba$$
    
    Subtracting the terms $a^2c+ac+bc+b^2c$ from the left-hand side and their equivalents $ca^2+ca+cb+cb^2$ from the right-hand side, we have:
    $$abc + bac = cab + cba$$ 
    for all $a, b, c \in R$.
    
    Let $c=a$ in the identity above, then
    $$aba + baa = aab + aba$$
    $$aba + ba^2 = a^2b + aba$$
    Cancelling $aba$ on both sides, we obtain:
    $$ba^2 = a^2b$$
    for all $a, b \in R$. The commutativity of $R$ follows since $(*)$ becomes $ax-xa=0$.
\end{solution}

\end{document}
