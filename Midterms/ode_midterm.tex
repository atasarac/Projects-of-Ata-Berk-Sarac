\documentclass{article}

% Language setting
% Replace `english' with e.g. `spanish' to change the document language
\usepackage[english]{babel}

% Set page size and margins
% Replace `letterpaper' with `a4paper' for UK/EU standard size
\usepackage[a4paper,top=2cm,bottom=2cm,left=3cm,right=3cm,marginparwidth=1.75cm]{geometry}

% Useful packages
\usepackage{amsmath}
\usepackage{amsthm}
\usepackage{amsfonts}
\usepackage{amssymb}
\usepackage{geometry}
\usepackage{graphicx}
\usepackage{calc}
\usepackage{xassoccnt}
\usepackage{fontenc}
\usepackage{mathtools}
\usepackage{tikz}
\usepackage{caption}
\usepackage{enumerate}
\usepackage{xcolor}
\usepackage{sectsty}
\usepackage{microtype}
\usepackage{titlesec}
\usepackage{etoolbox}
\usepackage{chngcntr}
\usepackage[inline]{enumitem}
\usepackage[colorlinks=true, allcolors=blue]{hyperref}


%\newtheorem
\theoremstyle{definition}
\newtheorem{definition}{Definition}[subsection]

\newtheorem{theorem}{Theorem}[subsection]
\newtheorem{lemma}{Lemma}[subsection]
\newtheorem{corollary}{Corollary}[subsection]

\newtheorem{example}{Example}[section]
\newtheorem{exercise}{Exercise}
\newtheorem{question}{Question}

\DeclareCoupledCountersGroup{theorems}
\DeclareCoupledCounters[name=theorems]{theorem,lemma}

\theoremstyle{remark}
\newtheorem{remark}{Remark}[section]
\newtheorem*{solution}{Solution}

%\newcommands
\newcommand{\R}{\mathbb{R}}
\newcommand{\C}{\mathbb{C}}
\newcommand{\Z}{\mathbb{Z}}
\newcommand{\Q}{\mathbb{Q}}
\newcommand{\N}{\mathbb{N}}
\newcommand{\F}{\mathbb{F}}
\newcommand{\calP}{\mathcal{P}}
\newcommand{\powerset}[1]{\mathscr{P}(#1)}
\newcommand{\Mod}[1]{\ (\mathrm{mod}\ #1)}
%%%

\title{ODE Midterm}
\author{Ata Berk Saraç}
\date{April 2025}

\begin{document}

\maketitle

\begin{question}
    Solve the following first order differential equations:
    \begin{enumerate}[label=(\alph*)]
        \item $\cos(x)dy=y(\sin(x)-y)dx,$
        \item $(2x+y-3)dy=(x+2y-3)dx.$
    \end{enumerate}
\end{question}

\begin{solution}
    \begin{enumerate}[label=(\alph*)]
        \item This D.E. turns into
        \[y'-\tan(x) y=-\sec(x) y^2,\]
        we recognize that it is a Bernoulli D.E.
        It follows that
        \[y^{-2}y'-\tan(x) y^{-1}=-\sec(x),\]
        substituting $v=y^{-1}$, we have $v'=-y^{-2}y'$ and so
        \[v'+\tan(x)v=\sec(x)\]
        which is a linear D.E.
        
        Consider a function $\mu(x)$ such that $\mu'(x)=\mu(x) \tan(x)$
        \[\mu(x) v'+\mu'(x) v=\mu(x) \sec(x).\]
        Therefore, $\mu(x)=e^{\int \tan(x)\,dx}$ so that $\mu(x)=e^{-\ln|\cos(x)|}=\sec(x)$.
        Thus, 
        \[v(x)=\frac{\int \sec^2(x)\, dx + c}{\sec(x)}=\frac{\tan(x)+c}{\sec(x)}=\sin(x)+c\cos(x)\]
        so that 
        \[y(x)=\frac{1}{\sin(x)+c\cos(x)}\]
        where $c$ is a constant.

        \item Let $u=x-1$ and $v=y-1$. Then
        $$\frac{dv}{du}=\frac{u+2v}{2u+v}$$

        Assume that $z=\frac{v}{u}$, then $v'=z+uz'$; therefore 
        $$u\frac{dz}{du}=\frac{1-z^2}{2+z}.$$
        It follows that
        $$\int \frac{1}{1+z}\,dz-\int \frac{-1}{1-z}\,dz-\frac12\int \frac{-2z}{1-z^2}\,dz=\int \frac{1}{u}\,du,$$
        thus
        $$\\ln|1+z|- \ln |1-z|-\frac12 \ln|1-z^2|=\ln\Bigl|\frac{1+z}{(1-z)^2}\Bigr|=\ln|u|+C,$$
        we get
        $$\frac{1}{1-z}=Cu,$$
        and finally
        $$z=1-\frac{1}{Cu}.$$

        Now, we rearrange the equation above
        $$\frac{y-1}{x-1}=1-\frac{1}{C(x-1)}$$
        to get
        $$y(x)=x-\frac{1}{C}$$
        where $C$ is an arbitrary constant.
    \end{enumerate}
\end{solution}

\begin{question}
    Show that $y=1-4x$ is a particular solution of $y'=y^2+8xy+16x^2-5$, and find its general solution.
\end{question}

\begin{solution}
    Call $y$, $y_p$; then $y_p=1-4x$ and $y'_p=-4$. We check the validity of $y_p$ as a solution:
    $$y'_p=y_p^2+8xy_p+16x^2-5=(1-4x)^2+8x(1-4x)+16x^2-5=1-8x+16x^2+8x-32x^2+16x^2-5=-4,$$
    therefore $y_p$ is in fact a particular solution for our D.E.

    Observe that $y'=y^2+8xy+16x^2-5$ is a Riccati D.E. which can be turned into the form $(y+4x)^2-5$. So, let $u=y+4x$. Since $u'=y'+4$, we have $\frac{du}{dx}=u^2-1$ and so $\frac{du}{u^2-1}=dx$ which is a seperable equation.

    It follows that 
    $$\int \frac{du}{u^2-1}=\frac12 \int \Bigl(\frac{1}{u-1}-\frac{1}{u+1}\Bigr)\,du=\frac12 \ln \Bigl|\frac{u-1}{u+1}\Bigr|=x+C,$$
    so that
    $$\frac{u-1}{u+1}=Ke^{2x}$$
    where $K=e^{2C}.$

    We have that
    $$u-1=Ke^{2x}(u+1),$$
    $$u(x)=\frac{1+K^{2x}}{1-K^{2x}},$$
    thus
    $$y(x)=\frac{1+K^{2x}}{1-K^{2x}}-4x,$$
    where $K\in\R$ is an arbitrary constant.
\end{solution}

\begin{question}
    Show that $y_1=3x^2-1$ is a solution. Find the general solution of
    $$(1-x^2)y''-2xy'+6y=0.\, (*)$$
\end{question}

\begin{solution}
    Note that $y'_1=6x$ and $y''_1=6$. It follows that
    $$(1-x^2)6-2x\cdot 6x+6(3x^2-1)=6-6x^2-12x^2+18x^2-6=0.$$
    We can rewrite $(*)$ as
    $$y''-\frac{2x}{1-x^2}y'+\frac{6}{1-x^2}y=0.$$

    Now, we look for a second solution $y_2(x)$ such that
    $$y_2(x)=v(x)y_1(x)$$
    where $v(x)$ is a first-order D.E.
    Compute
    $$y'_2=v'y_1+vy'_1,\qquad y''_2=v''y_1+2(v'y'_1)+vy''_1;$$
    therefore
    $$\bigl[v''y_1+2(v'y'_1)+vy''_1\bigr]-\frac{2x}{1-x^2}\bigl[v'y_1+vy'_1+6xv\bigr]+\frac{6}{1-x^2}\bigl[vy_1\bigr]=0.$$
    Since $y_1$ itself satisfies $(*)$, all the terms proportional to $v$ cancel out so letting $P(x)=\frac{2x}{1-x^2}$, we are left with
    $$\bigl[v''y_1+2(v'y'_1)\bigr]-P\bigl[v'y_1\bigr]=0.$$
    
    Set $w(x)=v'(x)$. The equation above turns into
    $$w'y_1+(y'_1-Py_1)w=0\implies \frac{dw}{dx}+\Bigl(\frac{y'_1}{y_1}-P\Bigr)w=0$$
    which is a first order linear D.E.
\end{solution}

\begin{question}
    Find an integrating factor in terms of $u=x^2+y^2$ for the differential equation and then solve it:
    $$(x^3+xy^2-y)dx+xdy=0.$$
\end{question}

\begin{solution}
    Note that $\frac{du}{dx} = 2x$ and $\frac{du}{dy} = 2y$.
    The equation is not exact since \(\frac{\partial M}{\partial y} = 2xy - 1\) and \(\frac{\partial N}{\partial x} = 1\).
    \begin{align*}
        \frac{\mu'(u)}{\mu(u)}&= \frac{\frac{\partial N}{\partial x} - \frac{\partial M}{\partial y}}{2yM - 2xN}\\
        &= \frac{1 - (2xy - 1)}{2y(x^3 + xy^2 - y) - 2x^2}\\
        &=\frac{2(1 - xy)}{2(xy - 1)(x^2 + y^2)}\\
        &= \frac{-1}{x^2 + y^2},
    \end{align*}
    thus
    $$\mu(u)=e^{-\int\frac{1}{u}\,du}=e^{-\ln|u|}=\frac{1}{u}=\frac{1}{x^2+y^2}.$$

    Multiplying by the integrating factor, the equation becomes:
    \[
    \left( x - \frac{y}{x^2 + y^2} \right)dx + \frac{x}{x^2 + y^2} dy = 0.
    \]
    Now, \(\frac{\partial M'}{\partial y} = \frac{\partial N'}{\partial x} = -\frac{x^2 - y^2}{(x^2 + y^2)^2}\), confirming exactness.
    Integrate \(M'\) with respect to \(x\) to get
    \[
    \frac{x^2}{2} - \arctan\left(\frac{x}{y}\right) + h(y).
    \]
    Differentiate with respect to \(y\) and solve for \(h(y)\) to find \(h(y)\) is constant. Thus, the solution is:  
    \[
    \frac{x^2}{2} - \arctan\left(\frac{x}{y}\right) = C.
    \]
\end{solution}

\begin{question}
    Find the orthogonal curves of the family $y=-x-1+ce^x\,(*)$.
\end{question}

\begin{solution}
    We have $\frac{dy}{dx}=ce^x-1\,(\cdot)$, adding $-(*)$ and $(\cdot)$ we get 
    $$\frac{dy}{dx}=x+y,$$
    $$
     \frac{dy}{dx}\Big|_{\rm orth}
     = -\frac{1}{\displaystyle \tfrac{dy}{dx}}
     = -\frac{1}{x+y},
     $$

    thus
    $$\frac{dx}{dy}=-x-y$$
    is orthogonal to the curves defined by $(*)$.

    Then,
    $$\frac{dx}{dy}+x=-y$$
    which is a first order linear D.E., it follows that the integrating factor
    $$\mu(y)=e^{\int 1\,dy}=e^y,$$
    multiply all terms by $\mu(y)$
    $$e^y\frac{dx}{dy}+e^yx=\frac{d}{dy}(x e^y)=-e^y y.$$
    Integrate with respect to $y$
    $$x e^y =-\int y e^y\, dy + K = (1-y)\, e^y + K$$
    so that 
    $$x=1-y+K e^{-y}$$
    is a family of curves orthogonal to the family $(*)$.
\end{solution}

\begin{question}
    Let $y_1,y_2,\ldots,y_n$ be $n$ solutions of the homogeneous linear $n$-th order differential equation on an interval $I$. Then the set of solutions is linearly independent on $I$ if and only if $W(y_1,y_2,\ldots,y_n)\ne 0$ for every $x$ in the interval.
\end{question}

\begin{solution}
    Suppose, to the contrary, that $W(y_1,y_2,\ldots,y_n)(x_0)\ne 0$ at some point $x_0\in I$ but the $y_i$ are linearly dependent. Then there exist constants $c_1,\dots,c_n$, not all zero, so that
    $$
    c_1y_1(x) + \cdots + c_ny_n(x)= 0
    $$
    on $I$.
     
    Differentiating $n-1$ times gives a homogeneous linear system
    $$
    \begin{bmatrix}
    y_1(x_0)&\cdots&y_n(x_0)\\
    y_1'(x_0)&\cdots&y_n'(x_0)\\
    \vdots&&\vdots\\
    y_1^{(n-1)}(x_0)&\cdots&y_n^{(n-1)}(x_0)
    \end{bmatrix}
    \begin{bmatrix}c_1\\\vdots\\c_n\end{bmatrix}
    = 0.
    $$
    Call the matrices on the left-hand side $W$ and $c$ respectively. A nontrivial solution $c$ exists only if $W$ has determinant zero, since if $W$ had nonzero determinant, then it would be invertible, so that $W^{-1}Wc=c=0$.  But its determinant is exactly $W(y_1,y_2,\ldots,y_n)(x_0)$, which we assumed nonzero, a contradiction! Thus, $W(y_1,y_2,\ldots,y_n)(x)\ne 0$ implies that the set of solutions is linearly independent on $I$.
    
    Now, assume that the set of solutions is linearly independent on $I$. It follows that there exist $a_i$ corresponding to each $y_i$ such that
    $$\sum_{i=1}^n {a_iy_i}=0,$$
    differentiating one times we get
    $$\sum_{i=1}^n {a_iy'_i}=0$$
    and so on. We thus obtain that all columns are linearly independent, meaning that the rank of the matrix is $n$, since row and column ranks are equal, row rank is also $n$ which implies that the determinant of this matrix must be nonzero. The result follows.
\end{solution}

\begin{question}
    Find the general solution of the following differential equations:
    \begin{enumerate}[label=(\alph*)]
        \item $y^{(4)}+8y''+16y=0,$
        \item $9y''+6y'+5y=0$.
    \end{enumerate}
\end{question}

\begin{solution}
    \begin{enumerate}[label=(\alph*)]
        \item Assume a solution of the form $y=e^{rx}$, then $y^{(i)}=r^ie^{rx}$ for $i\in \N$. Substituting, our equation becomes
        $$r^4e^{rx}+8r^2e^{rx}+16e^{rx}=0,$$
        since $e^{rx}\ne 0$, we get
        $$r^4+8r^2+16=0.$$

        Now, let $u=r^2$. The equation becomes
        $$u^2+8u+16=(u+4)^2=0.$$
        So, $u=-4$ is a repeated root of multiplicity 2, and so $r=\pm\sqrt{-4}=\pm2i$, following that $r=+2i$ and $r=-2i$ are both repeated roots of multiplicity 2. The repeated roots \( r = \pm 2i \) (each with multiplicity 2) lead to the general solution:
        \[
        y(x) = (C_1 + C_2 x)\cos(2x) + (C_3 + C_4 x)\sin(2x),
        \]
        where \( C_1, C_2, C_3, C_4 \) are constants.

        \item For the differential equation \(9y'' + 6y' + 5y = 0\), we solve the characteristic equation \( 9r^2 + 6r + 5 = 0 \). Using the quadratic formula:
        \[
        r = \frac{-6 \pm \sqrt{36 - 180}}{18} = \frac{-6 \pm 12i}{18} = -\frac{1}{3} \pm \frac{2}{3}i.
        \]
        The general solution with complex roots \( \alpha \pm \beta i \) is:
        \[
        (x) = e^{-\frac{x}{3}} \left( C_1 \cos\left(\frac{2x}{3}\right) + C_2 \sin\left(\frac{2x}{3}\right) \right),
        \]
        where \( C_1, C_2 \) are constants.
    \end{enumerate}
\end{solution}

\end{document}
